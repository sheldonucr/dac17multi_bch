\section{Accurate analytical modeling for hydrostatic stress evolution}
\label{sec:multi_segment}

In this section, we present the new analytic solution and compact
models for EM-induced stress evolution in confined multi-segment
interconnect wires as shown in Fig.~\ref{fig:4seg-demo}. To facilitate
our analysis, we are looking at an abstract general multi-segment
interconnect structure with one continuous metallization layer shown in
Fig.\ref{fig:interconnect_tree}, which represents a long metal wire
with $n$ segments and $n+1$ terminals (vias) (presented by blue blocks)
connecting different metal layers.  Since many complex interconnect
structures such as power grids could be decomposed hierarchically into
smaller interconnect tree components like this structure, the accurate
analytic closed-form expressions describing hydrostatic stress
evolution for multi-branch interconnect trees will lay the groundwork
for the EM reliability analysis of those global interconnect networks.

\label{sec:analytical_stress}
\begin{figure}[ht] \centering
\includegraphics[width=80mm]{Sn.eps}
\caption{An example of general realistic interconnect structure.}
  \label{fig:interconnect_tree}
  \vspace{-0.12in}
\end{figure}

\subsection{The analytic solution for stress evolution for multi-segment interconnect wires}
The stress evolution process for each segment in
Fig.\ref{fig:interconnect_tree} can be described by a single
Korhonen's equation as shown in \eqref{eq:basic_em_s}. But solving
such equation directly is more difficult.  Instead, in our approach,
we decouple the original single Korhonen's equation into $n$ decoupled
Korhonen equations and each of them governing one segment of the
multi-segment wire. The partial differential equations (PDE) are
linked by the boundary conditions representing the continuity of
stress and fluxes at the junctions in the long metal
wire. Specifically, the PDE in \eqref{eq:basic_em_s} can re-written
into a number of decoupled PDEs as follows:
\begin{equation} \label{eq:general_interconnect_tree}\small
\begin{split}
&\frac{\partial \sigma_1}{\partial t}=\frac{\partial }{\partial
x}[\kappa_1(\frac{\partial \sigma_1}{\partial x}+G_1)],\;l_0\leq x \leq l_1 \\
&\frac{\partial \sigma_2}{\partial t}=\frac{\partial }{\partial
x}[\kappa_2(\frac{\partial \sigma_2}{\partial
x}+G_2)],\;l_1\leq x \leq l_2, \\
&\cdots\cdots\\
&\frac{\partial \sigma_{n-1}}{\partial t}=\frac{\partial }{\partial
x}[\kappa_{n-1}(\frac{\partial \sigma_{n-1}}{\partial
x}+G_{n-1})],\;l_{n-2}\leq x \leq l_{n-1}, \\
&\frac{\partial \sigma_{n}}{\partial t}=\frac{\partial }{\partial
x}[\kappa_{n}(\frac{\partial \sigma_{n}}{\partial
x}+G_{n})],\;l_{n-1}\leq x \leq l_n.
 \end{split}
 \end{equation}
 In \eqref{eq:general_interconnect_tree}, $\kappa_{i}$ and $G_i$
 ($i=1,2,\cdots,n$) are the stress diffusivity and the EM driving
 force for each segment wire. Boundary conditions for these equations
 are given as follows:
 \begin{equation} \label{bc:general_interconnect_tree}\small
\begin{split}
&\kappa_1(\frac{\partial \sigma_1}{\partial x}+G_1)=0,\;x=l_0,\\
&\sigma_1=\sigma_2,\;x=l_1,\\
&\kappa_1(\frac{\partial \sigma_1}{\partial
x}+G_1)=\kappa_2(\frac{\partial \sigma_2}{\partial
x}+G_2),\;x=l_1,\\
&\sigma_2=\sigma_3,\;x=l_2,\\
&\kappa_2(\frac{\partial \sigma_2}{\partial
x}+G_2)=\kappa_3(\frac{\partial \sigma_3}{\partial
x}+G_3),\;x=l_2,\\
&\cdots\cdots \\
&\kappa_n(\frac{\partial \sigma_n}{\partial
x}+G_n)=0,\;x=l_n.
 \end{split}
 \end{equation}
 Initial conditions for the void nucleation phase are given by
 $\sigma_i=0$ at $t=0$ at each segment.

 To solve the resulting decoupled equations in
 \eqref{eq:general_interconnect_tree} subject to the boundary
 conditions \eqref{bc:general_interconnect_tree}, the Laplace
 transformation technique is used to convert the PDE into the Laplace
 domain. For the sake of simplicity, we assume that
 $\kappa_1=\kappa_2=\cdots=\kappa_n$ and the EM driving force $G_i$ does not
 depend on the time $t$ and the position $x$. After transforming
 \eqref{eq:general_interconnect_tree}-\eqref{bc:general_interconnect_tree}
 by the Laplace transformation technique, we get a system of ordinary
 differential equations (ODEs):
  \begin{equation} \label{eq:multi_segment_ode}\small
\begin{split}
&\frac{d^2\hat{\sigma}_1(x,s)}{dx^2}=\frac{s}{\kappa_1}\hat{\sigma}_1(x,s),\;l_0\leq x\leq l_1,\\
&\frac{d^2\hat{\sigma}_2(x,s)}{dx^2}=\frac{s}{\kappa_2}\hat{\sigma}_2(x,s),\;l_1\leq x\leq l_2,\\
&\cdots\cdots\\
&\frac{d^2\hat{\sigma}_{n-1}(x,s)}{dx^2}=\frac{s}{\kappa_{n-1}}\hat{\sigma}_{n-1}(x,s),\;l_{n-2}\leq x\leq l_{n-1},\\
&\frac{d^2\hat{\sigma}_{n}(x,s)}{dx^2}=\frac{s}{\kappa_{n}}\hat{\sigma}_{n}(x,s),\;l_{n-1}\leq x\leq l_{n},
 \end{split}
 \end{equation}
where $\hat{\sigma}_{i}(x,s)=\int_0^{+\infty}e^{-st}\sigma_{i}(x,t)dt\;(i=1,2,\cdots,n)$ is the Laplace transform of $\sigma_{i}(x,t)$.
By using the Laplace transform of the boundary conditions \eqref{bc:general_interconnect_tree},
the analytical solution $\hat{\sigma}_{i}(x,s)$ of \eqref{eq:multi_segment_ode} can be obtained in the Laplace domain. Due to limit space,
we omit the derivation process of obtaining $\hat{\sigma}_{i}(x,s)$.

By taking the inverse Laplace transform of $\hat{\sigma}_{i}(x,s)$, we
can obtain the analytical solution $\sigma_i(x,t)$ for each PDE in
\eqref{eq:general_interconnect_tree}. Before we present the concrete
expression of $\sigma_i(x,t)$, we introduce the following notations:
\begin{equation} \label{generalNotations}\small
\begin{split}
&\xi_{n,1}^{i}(m,x)=l_n+2m(l_n-l_0)-x,\\
&\xi_{n,2}^{i}(m,x)=(2l_n-l_0)+2m(l_n-l_0)-x,\\
&\xi_{n,2p+1}^{i}(m,x)=(2l_n-l_p)+2m(l_n-l_0)-x,\\
&\xi_{n,2p+2}^{i}(m,x)=\left\{
   \begin{aligned}
   &(2l_n-2l_0+l_p)+2m(l_n-l_0)-x,\;\\
   &\qquad\qquad\qquad\qquad\qquad\quad 1\leq p<i,  \\
   &l_p+2m(l_n-l_0)-x,\;i\leq p\leq n-1, \\
      \end{aligned}
   \right. \\
&\eta_{n,1}^{i}(m,x)=(l_n-2l_0)+2m(l_n-l_0)+x,\\
&\eta_{n,2}^{i}(m,x)=-l_0+2m(l_n-l_0)+x,\\
&\eta_{n,2p+1}^{i}(m,x)=(l_p-2l_0)+2m(l_n-l_0)+x,\\
&\eta_{n,2p+2}^{i}(m,x)=\left\{
   \begin{aligned}
   &-l_p+2m(l_n-l_0)+x,\;1\leq p<i,  \\
   &(2l_n-l_p-2l_0)+2m(l_n-l_0)+x,\; \\
   &\qquad\qquad\qquad\qquad\qquad i\leq p \leq n-1, \\
   \end{aligned}
   \right.
\end{split}
\end{equation}
where $m$ is a nonnegative integer, $n$ is the number of segments, $i$ $(i=1,2,\cdots,n)$ is the index of current segment, and $p$ is an integer from $1$ to $n-1$.

Similar to the analytical method introduced in
\cite{ChenHuang:DAC'15}, we first introduce the following basic
function:
\begin{equation} \label{general_basisFunc}
g(x,t)=2\sqrt{\frac{\kappa t}{\pi}}e^{-\frac{x^2}{4\kappa
t}}-x\times\texttt{erfc}\{\frac{x}{2\sqrt{\kappa t}}\}.
\end{equation}
where the complementary error function $\texttt{erfc}\{x\}$ is defined
as
$\texttt{erfc}\{x\}=\frac{2}{\sqrt{\pi}}\int_x^{+\infty}e^{-t^2}dt$.

With those definitions, it turns out that a general form of analytical
solutions of stress evolution equations for the $i$-th
$(i=1,2,\cdots,n)$ segment can be derived as follows
\begin{equation} \label{eq:general_solution}\small
\begin{split}
&\sigma_{i}(x,t)=-\sum\limits_{m=0}^{+\infty}\{G_ng(\xi_{n,1}^{i}(m,x),t)-G_1g(\xi_{n,2}^{i}(m,x),t)\\
&-\sum\limits_{p=1}^{n-1}\frac{G_{p+1}-G_p}{2}(g(\xi_{n,2p+1}^{i}(m,x),t)+g(\xi_{n,2p+2}^{i}(m,x),t))\}\\
&-\sum\limits_{m=0}^{+\infty}\{G_ng(\eta_{n,1}^{i}(m,x),t)-G_1g(\eta_{n,2}^{i}(m,x),t)\\
&-\sum\limits_{p=1}^{n-1}\frac{G_{p+1}-G_p}{2}(g(\eta_{n,2p+1}^{i}(m,x),t)+g(\eta_{n,2p+2}^{i}(m,x),t))\}.
 \end{split}
 \end{equation}

\subsection{The compact EM model and time to failure calculation}
 We notice that the solution in \eqref{eq:general_solution} contains
 an infinite number of terms. One observation we have with Laplace
 transformation based solution with the infinite number of terms is that
 we may just need a few dominant terms for sufficient accuracy.
 Fortunately, we show in experimental section that this is indeed a
 case for our problem. For instance, we just keep the first dominant
 term ($m=0$), then the solution for the $i$-th $(i=1,2,\cdots,n)$
 segment, can be re-written as:
\begin{equation} \label{eq:approximate_solution}\small
\begin{split}
&\sigma_{i}(x,t)\approx-\{G_ng(\xi_{n,1}^{i}(0,x),t)-G_1g(\xi_{n,2}^{i}(0,x),t)\\
&-\sum\limits_{p=1}^{n-1}\frac{G_{p+1}-G_p}{2}(g(\xi_{n,2p+1}^{i}(0,x),t)+g(\xi_{n,2j+2}^{i}(0,x),t))\}\\
&-\{G_ng(\eta_{n,1}^{i}(0,x),t)-G_1g(\eta_{n,2}^{i}(0,x),t)\\
&-\sum\limits_{p=1}^{n-1}\frac{G_{p+1}-G_p}{2}(g(\eta_{n,2p+1}^{i}(0,x),t)+g(\eta_{n,2p+2}^{i}(0,x),t))\}.
 \end{split}
 \end{equation}
 By just use a few dominant terms, we can obtain the very compact EM
 stress development models for the multi-segment interconnect wires.
 It should be noted that when the wire segment stress
 $\sigma_{i}(x,t)$ reaches a critical stress $\sigma_{crit}$, then the
 void nucleation time $t_{nuc}$ can be calculated for each segment by
 solving the equation $\sigma_{i}(x,t_{nuc})=\sigma_{crit}$.

 % It should be noted that the present analysis method in analytical
 % characterization is similar to what has been reported in our early
 % work of analytical modeling of electromigration for multi-branch
 % interconnect tree. However, the purpose of this paper is not to
 % develop any new analysis method, but to illustrate the closed-form
 % expression for the solution of stress evolution equations for more
 % complex interconnect trees embedded in the frequently employed
 % circuits.


 % After transforming all the equations and the corresponding BC into
 % Laplace space we get a system of $n$ equations similar to
 % \eqref{eq:dotted_I_tree_lap} and BC represented by $n+1$ equations
 % similar to \eqref{bc:dotted_I_tree_lap}. Solution of each ODE
 % represented by \eqref{eq:dotted_I_tree_fre} with two unknown
 % parameters, which will be found from BC, similarly to what was done
 % in the case of two-segments/three terminals case,
 % \eqref{cofficients_I}. A problem that we are solving is
 % determination of the moment of time and location on the
 % $\mathrm{V_{DD}}$ rail when and where the first void will be
 % nucleated. To do this we do not need to make the inverse Laplace
 % transformation. It is enough to compare the stress in the Laplace
 % space $\sigma_i(x,s)$ with the Laplace transform of the critical
 % stress. It should be also taken into account that not all junctions
 % but just some of them can provide a condition for void nucleation:
 % $\sigma=\sigma_{crit}$. Basically just two types of junctions
 % characterized by specific configurations of the current directions
 % should be considered. First type is the terminating segments
 % serving as the current outlets (electron cathodes), and second the
 % junctions separating two segments with electron flows directed
 % outward this junctions.





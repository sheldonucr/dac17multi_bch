\documentclass[twocolumn,journal]{IEEEtran}

\usepackage{graphicx}
\usepackage{subfigure}
\usepackage{amsfonts}
\usepackage{amsmath}
\usepackage{multirow}
\usepackage{url}


%\usepackage[usenames,dvips]{color}
%\usepackage{graphicx}
%\usepackage{algorithm}
%\usepackage{algorithmic}
%\usepackage{multirow}
%\usepackage{subfig}
\usepackage{psfig}
%\usepackage{booktabs}
\usepackage{epstopdf}
%\usepackage{graphicx}
%\usepackage[usenames]{color}
%\usepackage{amssymb,amsmath}
%\usepackage{authblk}
%\usepackage{url}
%% \usepackage{mathdots}
\usepackage{arydshln}
\usepackage{pifont}
\usepackage{cite}

\usepackage{color}




\newtheorem{theory}{Theorem}
\newtheorem{define}{Definition}
\newtheorem{lemma}{Lemma}
\newtheorem{corollary}{Corollary}
\newtheorem{proposition}{Proposition}


\renewcommand{\baselinestretch}{0.99}

\hyphenation{op-tical net-works semi-conduc-tor}

\begin{document}
%\title{Physics-Based Analytical Modeling for Electromigration Reliability in Multi-Branch Interconnect Trees Considering Time-Varying Temperature}
\title{Physics-Based Analytical Modeling of Electromigration Reliability for Multi-Branch Interconnect Trees}
%\title{Dynamic Temperature-Aware Reliability Modeling for Multi-Branch Interconnect Trees}


\author{
Jiangtao Peng, Hai-Bao Chen, Taeyoung Kim, Hengyang Zhao, Sheldon X.-D. Tan

\thanks{
%\textbf{Jiangtao Peng} is studying in the Department of Micro/Nano-electronics, Shanghai Jiao Tong University, Shanghai, China. She is devoted to electromigration reliability and smart building now.
%
%\textbf{Hai-Bao Chen} received the B.S. degree in information and computing sciences, and the M.S. and Ph.D. degrees in applied mathematics from Xi��an Jiaotong University, Xi��an, China, in 2006, 2008, and 2012, respectively. He then joined Huawei Technologies, where he focused on cloud computing and big data. He was a Post-Doctoral Research Fellow with Electrical Engineering Department, University of California, Riverside, Riverside, CA, USA, from 2013 to 2014. He is currently an Assistant Professor in the Department of Micro/Nano-electronics, Shanghai Jiao Tong University, Shanghai, China. His current research interests include model order reduction, system and control theory, circuit simulation, cloud computing and big data, and electromigration reliability. Dr. Chen has authored or co-authored more than 25 papers in scientific journals and conference proceedings. He received one Best Paper Award nomination from Asia and South Pacific Design Automation Conference (ASP-DAC) in 2015.
} }

% make the title area
\maketitle


\begin{abstract}




%Thermal effects has become a recent major research for being a limiting factor in high performance circuit design. Due to the strong thermal-dependence of leakage power, circuit performance, IC package cost and reliability, changing temperature has been an emerging concern on electromigration (EM) in conducting metal lines with complex IC structures. However, an average temperature in the EM failure process is commonly used in most existing EM models and analytic techniques considering dynamic temperature are mainly focused on a single metal wire. In this paper, we propose an analytic method to calculate the stress evolution considering time-varying temperature effects during the void nucleation phase for some multi-branch interconnect trees, including the straight-line there-terminal wires, the T-shaped four-terminal wires and the cross-shaped five-terminal wires. The proposed closed-form expression can be used to calculate the hydrostatic stress evolution with time-varying temperature. Experiment results show that the obtained analytic solutions match well with the numerical results calculated using COMSOL and thus the proposed models can be used in traditional EM reliability analysis tools.

In high performance circuit design, thermal effect on electromigration (EM) reliability has become a recent major research for being a limiting factor. Due to the strong thermal-dependence of leakage power, circuit performance, IC package cost and reliability, considering time-varying temperature effect has been an emerging concern on EM in conducting metal lines with complex interconnect structures. Dynamic temperature-aware reliability modeling has been proposed as a tool to explore the trade-off between the accuracy of lifetime prediction and the system performance. However, average temperature in the EM failure process is commonly used in most existing EM models and analytic techniques considering time-varying temperature are mainly focused on a single metal wire. In this paper, we propose a novel analytic method to calculate the stress evolution considering time-varying temperature effects during the void nucleation phase for some multi-branch interconnect trees, including the straight-line there-terminal wires, the T-shaped four-terminal wires and the cross-shaped five-terminal wires. The proposed closed-form expression can be used to calculate the hydrostatic stress evolution with time-varying temperature. Experiment results show that the obtained analytic solutions match well with the numerical results calculated using COMSOL and thus the proposed models can be used in traditional EM reliability analysis tools.

%Dynamic reliability management
%(DRM) has been proposed as a mechanism to dynamically explore the tradeoff
%between system performance and reliability margin. However, existing
%DRM methods are hampered by the fact that they do not accurately model
%spatial and temporal variations in process and temperature parameters
%which have a strong impact on chip reliability. In addition, they make
%the simplifying assumption that the future workloads are identical to the
%currently observed one. This makes them sensitive to sudden workload
%variations and outliers. In this paper, we present a novel workloadaware
%dynamic reliability management framework that accounts for local
%variations in both the process and temperature. The reliability estimation,
%along with the predicted remaining workload is fed to a dynamic voltage/frequency
%scaling module to manage the system reliability and optimize
%processor performance. Using a fast on-line analytical/table-look-up method
%we demonstrate an average error of 1% with up to 5 orders of magnitude
%speedup compared to Monte Carlo simulation. Experiments on an Alphalike
%processor show our DRM framework fully utilizes the available margin
%and achieves 28.7% performance improvement on average.
\end{abstract}

\vspace{0.15in}
%\begin{keywords}
%electromigration, multi-branch interconnect trees, analytical model, time-varying temperature, stress evolution.
%\end{keywords}


\input introduction.tex
\input temperature_effect.tex
\input solution_multi_branch.tex
\input analytical_expressions.tex
\input dynamic_modeling.tex
\input experimental_results.tex




\section{Conclusion}
\label{sec:conclusion}

In this paper, we have proposed a new analysis methods for multi-terminal interconnect tree considering changing temperature. The new time-varying model show an excellent agreement with the detailed numerical analysis obtained from COMSOL.




\bibliographystyle{ieeetr}

\bibliography{../../bib/reliability,../../bib/reliability_papers,../../bib/stochastic,../../bib/simulation,../../bib/modeling,../../bib/reduction,../../bib/misc,../../bib/architecture,../../bib/mscad_pub,../../bib/thermal_power}
%\bibliography{../../../bib/reliability.bib,reliability_papers,../../../bib/stochastic,../../../bib/simulation,../../../bib/modeling,../../../bib/reduction,../../../bib/misc,../../../bib/architecture,../../../bib/mscad_pub,../../../bib/thermal_power}


\end{document}

%\documentclass[twocolumn,journal]{IEEEtran}
\documentclass[twocolumn,conference]{IEEEtran}

\usepackage{graphicx}
\usepackage{subfigure}
\usepackage{amsfonts}
\usepackage{amsmath}
\usepackage{multirow}
\usepackage{url}


%\usepackage[usenames,dvips]{color}
%\usepackage{graphicx}
%\usepackage{algorithm}
%\usepackage{algorithmic}
%\usepackage{multirow}
%\usepackage{subfig}
\usepackage{psfig}
%\usepackage{booktabs}
\usepackage{epstopdf}
%\usepackage{graphicx}
%\usepackage[usenames]{color}
%\usepackage{amssymb,amsmath}
%\usepackage{authblk}
%\usepackage{url}
%% \usepackage{mathdots}
\usepackage{arydshln}
\usepackage{pifont}
\usepackage{cite}

\usepackage{color}




\newtheorem{theory}{Theorem}
\newtheorem{define}{Definition}
\newtheorem{lemma}{Lemma}
\newtheorem{corollary}{Corollary}
\newtheorem{proposition}{Proposition}


\renewcommand{\baselinestretch}{0.96}

\hyphenation{op-tical net-works semi-conduc-tor}

\begin{document}
%\title{Physics-Based Analytical Modeling for Electromigration Reliability in Multi-Branch Interconnect Trees Considering Time-Varying Temperature}
\title{Physics-Based Analytical Modeling of Electromigration Reliability for Multi-Segment Interconnect Wires}
%\title{Dynamic Temperature-Aware Reliability Modeling for Multi-Branch Interconnect Trees}


%\author{
%Jiangtao Peng, Hai-Bao Chen, Taeyoung Kim, Hengyang Zhao, Sheldon X.-D. Tan
%
%\thanks{
%%\textbf{Jiangtao Peng} is studying in the Department of Micro/Nano-electronics, Shanghai Jiao Tong University, Shanghai, China. She is devoted to electromigration reliability and smart building now.
%%
%%\textbf{Hai-Bao Chen} received the B.S. degree in information and computing sciences, and the M.S. and Ph.D. degrees in applied mathematics from Xi��an Jiaotong University, Xi��an, China, in 2006, 2008, and 2012, respectively. He then joined Huawei Technologies, where he focused on cloud computing and big data. He was a Post-Doctoral Research Fellow with Electrical Engineering Department, University of California, Riverside, Riverside, CA, USA, from 2013 to 2014. He is currently an Assistant Professor in the Department of Micro/Nano-electronics, Shanghai Jiao Tong University, Shanghai, China. His current research interests include model order reduction, system and control theory, circuit simulation, cloud computing and big data, and electromigration reliability. Dr. Chen has authored or co-authored more than 25 papers in scientific journals and conference proceedings. He received one Best Paper Award nomination from Asia and South Pacific Design Automation Conference (ASP-DAC) in 2015.
%} }

% make the title area
\maketitle


\begin{abstract}

  Electromigration (EM) is the major concern for the VLSI back end of
  the line (BEOL) reliability. For EM modeling and assessment, one
  important problem is to perform fast EM time to failure analysis for
  practical interconnect layouts such as clock and power/ground
  networks consisting of many multi-segment wires.  However, existing
  EM modeling and analysis techniques are mainly developed for a
  single wire. Although there exist some early effects to resolve this
  issue, but no general analytic and compact models have been
  developed to estimate the EM-induced stress evolution for predicting
  the time to failure. In this paper, we propose a new analytic
  expression to calculate hydrostatic stress evolution for general
  multi-segment interconnect wire with multiple terminals. The new
  method is based on the Laplace transformation method on the
  Korhonen's equation with blocked atom flux boundary conditions. The
  analytical solutions in terms of a set of auxiliary basis functions
  using the complementary error function agree well with the numerical
  analysis results. We further demonstrates that using the first one
  dominant basis function approximation can lead to 2\% error, which
  leads to very compact EM models with sufficient accuracy.


\end{abstract}

% \vspace{0.15in}
% \begin{keywords}
% electromigration, multi-branch interconnect trees, analytical model, time-varying temperature, stress evolution.
% \end{keywords}


\input introduction.tex

% the EM model review
\input reliability_modeling.tex

% the new analytic solution
\input solution_multi_branch.tex

%The results
\input experimental_results.tex

\section{Conclusion}
\label{sec:conclusion}
In this paper, we have proposed a new analysis method for EM stress
evolution during the void nucleation phase for multi-segment wires
commonly seen in practical VLSI layouts, which can be used for
practical VLSI interconnect design techniques.  An exact series
solution to the stress evolution expression for multi-segment wires
has been derived by using the Laplace transformation technique. The
obtained physics-based analytic EM model results are in good agreement
with the calculation by the finite element analysis tool
COMSOL. Experiment results demonstrate that using the first dominant
term approximation of the exact series solution can lead to around 2\%
error, which leads to very compact EM models with sufficient accuracy.

\bibliographystyle{ieeetr}

\bibliography{../../bib/reliability.bib,../../bib/reliability_papers,../../bib/stochastic,../../bib/simulation,../../bib/modeling,../../bib/reduction,../../bib/misc,../../bib/architecture,../../bib/mscad_pub,../../bib/thermal_power}


%\bibliography{../../bib/reliability.bib,reliability_papers,../../bib/stochastic,../../bib/simulation,../../bib/modeling,../../bib/reduction,../../bib/misc,../../bib/architecture,../../bib/mscad_pub,../../bib/thermal_power}
%\bibliography{../../../bib/reliability.bib,reliability_papers,../../../bib/stochastic,../../../bib/simulation,../../../bib/modeling,../../../bib/reduction,../../../bib/misc,../../../bib/architecture,../../../bib/mscad_pub,../../../bib/thermal_power}


\end{document}
